
% Default to the notebook output style

    


% Inherit from the specified cell style.




    
\documentclass{article}
\usepackage{parskip}


\AtBeginDocument{
\heavyrulewidth=.08em
\lightrulewidth=.05em
\cmidrulewidth=.03em
\belowrulesep=.65ex
\belowbottomsep=0pt
\aboverulesep=.4ex
\abovetopsep=0pt
\cmidrulesep=\doublerulesep
\cmidrulekern=.5em
\defaultaddspace=.5em
}
\newcommand{\fudm}[2]{\frac{\mathrm{D} #1}{\mathrm{D} #2}}
\newcommand{\pad}[2]{\frac{\partial #1}{\partial #2}}
\newcommand{\ppad}[2]{\frac{\partial^2 #1}{\partial #2^2}}
\newcommand{\ppadd}[3]{\frac{\partial^2 #1}{\partial #2 \partial #3}}
\newcommand{\nnabla}{\nabla^2}
\newcommand{\eps}{\epsilon}
\newcommand{\vdetail}[1]{\vb{#1}=\begin{pmatrix}#1_1\\#1_2\\#1_3\end{pmatrix}}
\newcommand{\vb}[1]{\mathbf{#1}}
\newcommand{\va}[1]{\vec{#1}}
\newcommand{\tb}[1]{\underline{\underline{\mathbf{#1}}}}
\newcommand{\fud}[2]{\frac{\mathrm{d} #1}{\mathrm{d} #2}}



    
    
    \usepackage{graphicx} % Used to insert images
    \usepackage{adjustbox} % Used to constrain images to a maximum size 
    \usepackage{color} % Allow colors to be defined
    \usepackage{enumerate} % Needed for markdown enumerations to work
    \usepackage{geometry} % Used to adjust the document margins
    \usepackage{amsmath} % Equations
    \usepackage{amssymb} % Equations
    \usepackage{eurosym} % defines \euro
    \usepackage[mathletters]{ucs} % Extended unicode (utf-8) support
    \usepackage[utf8x]{inputenc} % Allow utf-8 characters in the tex document
    \usepackage{fancyvrb} % verbatim replacement that allows latex
    \usepackage{grffile} % extends the file name processing of package graphics 
                         % to support a larger range 
    % The hyperref package gives us a pdf with properly built
    % internal navigation ('pdf bookmarks' for the table of contents,
    % internal cross-reference links, web links for URLs, etc.)
    \usepackage{hyperref}
    \usepackage{longtable} % longtable support required by pandoc >1.10
    \usepackage{booktabs}  % table support for pandoc > 1.12.2
    \usepackage{ulem} % ulem is needed to support strikethroughs (\sout)
    

    
    
    \definecolor{orange}{cmyk}{0,0.4,0.8,0.2}
    \definecolor{darkorange}{rgb}{.71,0.21,0.01}
    \definecolor{darkgreen}{rgb}{.12,.54,.11}
    \definecolor{myteal}{rgb}{.26, .44, .56}
    \definecolor{gray}{gray}{0.45}
    \definecolor{lightgray}{gray}{.95}
    \definecolor{mediumgray}{gray}{.8}
    \definecolor{inputbackground}{rgb}{.95, .95, .85}
    \definecolor{outputbackground}{rgb}{.95, .95, .95}
    \definecolor{traceback}{rgb}{1, .95, .95}
    % ansi colors
    \definecolor{red}{rgb}{.6,0,0}
    \definecolor{green}{rgb}{0,.65,0}
    \definecolor{brown}{rgb}{0.6,0.6,0}
    \definecolor{blue}{rgb}{0,.145,.698}
    \definecolor{purple}{rgb}{.698,.145,.698}
    \definecolor{cyan}{rgb}{0,.698,.698}
    \definecolor{lightgray}{gray}{0.5}
    
    % bright ansi colors
    \definecolor{darkgray}{gray}{0.25}
    \definecolor{lightred}{rgb}{1.0,0.39,0.28}
    \definecolor{lightgreen}{rgb}{0.48,0.99,0.0}
    \definecolor{lightblue}{rgb}{0.53,0.81,0.92}
    \definecolor{lightpurple}{rgb}{0.87,0.63,0.87}
    \definecolor{lightcyan}{rgb}{0.5,1.0,0.83}
    
    % commands and environments needed by pandoc snippets
    % extracted from the output of `pandoc -s`
    \providecommand{\tightlist}{%
      \setlength{\itemsep}{0pt}\setlength{\parskip}{0pt}}
    \DefineVerbatimEnvironment{Highlighting}{Verbatim}{commandchars=\\\{\}}
    % Add ',fontsize=\small' for more characters per line
    \newenvironment{Shaded}{}{}
    \newcommand{\KeywordTok}[1]{\textcolor[rgb]{0.00,0.44,0.13}{\textbf{{#1}}}}
    \newcommand{\DataTypeTok}[1]{\textcolor[rgb]{0.56,0.13,0.00}{{#1}}}
    \newcommand{\DecValTok}[1]{\textcolor[rgb]{0.25,0.63,0.44}{{#1}}}
    \newcommand{\BaseNTok}[1]{\textcolor[rgb]{0.25,0.63,0.44}{{#1}}}
    \newcommand{\FloatTok}[1]{\textcolor[rgb]{0.25,0.63,0.44}{{#1}}}
    \newcommand{\CharTok}[1]{\textcolor[rgb]{0.25,0.44,0.63}{{#1}}}
    \newcommand{\StringTok}[1]{\textcolor[rgb]{0.25,0.44,0.63}{{#1}}}
    \newcommand{\CommentTok}[1]{\textcolor[rgb]{0.38,0.63,0.69}{\textit{{#1}}}}
    \newcommand{\OtherTok}[1]{\textcolor[rgb]{0.00,0.44,0.13}{{#1}}}
    \newcommand{\AlertTok}[1]{\textcolor[rgb]{1.00,0.00,0.00}{\textbf{{#1}}}}
    \newcommand{\FunctionTok}[1]{\textcolor[rgb]{0.02,0.16,0.49}{{#1}}}
    \newcommand{\RegionMarkerTok}[1]{{#1}}
    \newcommand{\ErrorTok}[1]{\textcolor[rgb]{1.00,0.00,0.00}{\textbf{{#1}}}}
    \newcommand{\NormalTok}[1]{{#1}}
    
    % Additional commands for more recent versions of Pandoc
    \newcommand{\ConstantTok}[1]{\textcolor[rgb]{0.53,0.00,0.00}{{#1}}}
    \newcommand{\SpecialCharTok}[1]{\textcolor[rgb]{0.25,0.44,0.63}{{#1}}}
    \newcommand{\VerbatimStringTok}[1]{\textcolor[rgb]{0.25,0.44,0.63}{{#1}}}
    \newcommand{\SpecialStringTok}[1]{\textcolor[rgb]{0.73,0.40,0.53}{{#1}}}
    \newcommand{\ImportTok}[1]{{#1}}
    \newcommand{\DocumentationTok}[1]{\textcolor[rgb]{0.73,0.13,0.13}{\textit{{#1}}}}
    \newcommand{\AnnotationTok}[1]{\textcolor[rgb]{0.38,0.63,0.69}{\textbf{\textit{{#1}}}}}
    \newcommand{\CommentVarTok}[1]{\textcolor[rgb]{0.38,0.63,0.69}{\textbf{\textit{{#1}}}}}
    \newcommand{\VariableTok}[1]{\textcolor[rgb]{0.10,0.09,0.49}{{#1}}}
    \newcommand{\ControlFlowTok}[1]{\textcolor[rgb]{0.00,0.44,0.13}{\textbf{{#1}}}}
    \newcommand{\OperatorTok}[1]{\textcolor[rgb]{0.40,0.40,0.40}{{#1}}}
    \newcommand{\BuiltInTok}[1]{{#1}}
    \newcommand{\ExtensionTok}[1]{{#1}}
    \newcommand{\PreprocessorTok}[1]{\textcolor[rgb]{0.74,0.48,0.00}{{#1}}}
    \newcommand{\AttributeTok}[1]{\textcolor[rgb]{0.49,0.56,0.16}{{#1}}}
    \newcommand{\InformationTok}[1]{\textcolor[rgb]{0.38,0.63,0.69}{\textbf{\textit{{#1}}}}}
    \newcommand{\WarningTok}[1]{\textcolor[rgb]{0.38,0.63,0.69}{\textbf{\textit{{#1}}}}}
    
    
    % Define a nice break command that doesn't care if a line doesn't already
    % exist.
    \def\br{\hspace*{\fill} \\* }
    % Math Jax compatability definitions
    \def\gt{>}
    \def\lt{<}
    % Document parameters
    \title{lecture 9}
    
    
    

    % Pygments definitions
    
\makeatletter
\def\PY@reset{\let\PY@it=\relax \let\PY@bf=\relax%
    \let\PY@ul=\relax \let\PY@tc=\relax%
    \let\PY@bc=\relax \let\PY@ff=\relax}
\def\PY@tok#1{\csname PY@tok@#1\endcsname}
\def\PY@toks#1+{\ifx\relax#1\empty\else%
    \PY@tok{#1}\expandafter\PY@toks\fi}
\def\PY@do#1{\PY@bc{\PY@tc{\PY@ul{%
    \PY@it{\PY@bf{\PY@ff{#1}}}}}}}
\def\PY#1#2{\PY@reset\PY@toks#1+\relax+\PY@do{#2}}

\expandafter\def\csname PY@tok@gd\endcsname{\def\PY@tc##1{\textcolor[rgb]{0.63,0.00,0.00}{##1}}}
\expandafter\def\csname PY@tok@gu\endcsname{\let\PY@bf=\textbf\def\PY@tc##1{\textcolor[rgb]{0.50,0.00,0.50}{##1}}}
\expandafter\def\csname PY@tok@gt\endcsname{\def\PY@tc##1{\textcolor[rgb]{0.00,0.27,0.87}{##1}}}
\expandafter\def\csname PY@tok@gs\endcsname{\let\PY@bf=\textbf}
\expandafter\def\csname PY@tok@gr\endcsname{\def\PY@tc##1{\textcolor[rgb]{1.00,0.00,0.00}{##1}}}
\expandafter\def\csname PY@tok@cm\endcsname{\let\PY@it=\textit\def\PY@tc##1{\textcolor[rgb]{0.25,0.50,0.50}{##1}}}
\expandafter\def\csname PY@tok@vg\endcsname{\def\PY@tc##1{\textcolor[rgb]{0.10,0.09,0.49}{##1}}}
\expandafter\def\csname PY@tok@vi\endcsname{\def\PY@tc##1{\textcolor[rgb]{0.10,0.09,0.49}{##1}}}
\expandafter\def\csname PY@tok@mh\endcsname{\def\PY@tc##1{\textcolor[rgb]{0.40,0.40,0.40}{##1}}}
\expandafter\def\csname PY@tok@cs\endcsname{\let\PY@it=\textit\def\PY@tc##1{\textcolor[rgb]{0.25,0.50,0.50}{##1}}}
\expandafter\def\csname PY@tok@ge\endcsname{\let\PY@it=\textit}
\expandafter\def\csname PY@tok@vc\endcsname{\def\PY@tc##1{\textcolor[rgb]{0.10,0.09,0.49}{##1}}}
\expandafter\def\csname PY@tok@il\endcsname{\def\PY@tc##1{\textcolor[rgb]{0.40,0.40,0.40}{##1}}}
\expandafter\def\csname PY@tok@go\endcsname{\def\PY@tc##1{\textcolor[rgb]{0.53,0.53,0.53}{##1}}}
\expandafter\def\csname PY@tok@cp\endcsname{\def\PY@tc##1{\textcolor[rgb]{0.74,0.48,0.00}{##1}}}
\expandafter\def\csname PY@tok@gi\endcsname{\def\PY@tc##1{\textcolor[rgb]{0.00,0.63,0.00}{##1}}}
\expandafter\def\csname PY@tok@gh\endcsname{\let\PY@bf=\textbf\def\PY@tc##1{\textcolor[rgb]{0.00,0.00,0.50}{##1}}}
\expandafter\def\csname PY@tok@ni\endcsname{\let\PY@bf=\textbf\def\PY@tc##1{\textcolor[rgb]{0.60,0.60,0.60}{##1}}}
\expandafter\def\csname PY@tok@nl\endcsname{\def\PY@tc##1{\textcolor[rgb]{0.63,0.63,0.00}{##1}}}
\expandafter\def\csname PY@tok@nn\endcsname{\let\PY@bf=\textbf\def\PY@tc##1{\textcolor[rgb]{0.00,0.00,1.00}{##1}}}
\expandafter\def\csname PY@tok@no\endcsname{\def\PY@tc##1{\textcolor[rgb]{0.53,0.00,0.00}{##1}}}
\expandafter\def\csname PY@tok@na\endcsname{\def\PY@tc##1{\textcolor[rgb]{0.49,0.56,0.16}{##1}}}
\expandafter\def\csname PY@tok@nb\endcsname{\def\PY@tc##1{\textcolor[rgb]{0.00,0.50,0.00}{##1}}}
\expandafter\def\csname PY@tok@nc\endcsname{\let\PY@bf=\textbf\def\PY@tc##1{\textcolor[rgb]{0.00,0.00,1.00}{##1}}}
\expandafter\def\csname PY@tok@nd\endcsname{\def\PY@tc##1{\textcolor[rgb]{0.67,0.13,1.00}{##1}}}
\expandafter\def\csname PY@tok@ne\endcsname{\let\PY@bf=\textbf\def\PY@tc##1{\textcolor[rgb]{0.82,0.25,0.23}{##1}}}
\expandafter\def\csname PY@tok@nf\endcsname{\def\PY@tc##1{\textcolor[rgb]{0.00,0.00,1.00}{##1}}}
\expandafter\def\csname PY@tok@si\endcsname{\let\PY@bf=\textbf\def\PY@tc##1{\textcolor[rgb]{0.73,0.40,0.53}{##1}}}
\expandafter\def\csname PY@tok@s2\endcsname{\def\PY@tc##1{\textcolor[rgb]{0.73,0.13,0.13}{##1}}}
\expandafter\def\csname PY@tok@nt\endcsname{\let\PY@bf=\textbf\def\PY@tc##1{\textcolor[rgb]{0.00,0.50,0.00}{##1}}}
\expandafter\def\csname PY@tok@nv\endcsname{\def\PY@tc##1{\textcolor[rgb]{0.10,0.09,0.49}{##1}}}
\expandafter\def\csname PY@tok@s1\endcsname{\def\PY@tc##1{\textcolor[rgb]{0.73,0.13,0.13}{##1}}}
\expandafter\def\csname PY@tok@ch\endcsname{\let\PY@it=\textit\def\PY@tc##1{\textcolor[rgb]{0.25,0.50,0.50}{##1}}}
\expandafter\def\csname PY@tok@m\endcsname{\def\PY@tc##1{\textcolor[rgb]{0.40,0.40,0.40}{##1}}}
\expandafter\def\csname PY@tok@gp\endcsname{\let\PY@bf=\textbf\def\PY@tc##1{\textcolor[rgb]{0.00,0.00,0.50}{##1}}}
\expandafter\def\csname PY@tok@sh\endcsname{\def\PY@tc##1{\textcolor[rgb]{0.73,0.13,0.13}{##1}}}
\expandafter\def\csname PY@tok@ow\endcsname{\let\PY@bf=\textbf\def\PY@tc##1{\textcolor[rgb]{0.67,0.13,1.00}{##1}}}
\expandafter\def\csname PY@tok@sx\endcsname{\def\PY@tc##1{\textcolor[rgb]{0.00,0.50,0.00}{##1}}}
\expandafter\def\csname PY@tok@bp\endcsname{\def\PY@tc##1{\textcolor[rgb]{0.00,0.50,0.00}{##1}}}
\expandafter\def\csname PY@tok@c1\endcsname{\let\PY@it=\textit\def\PY@tc##1{\textcolor[rgb]{0.25,0.50,0.50}{##1}}}
\expandafter\def\csname PY@tok@o\endcsname{\def\PY@tc##1{\textcolor[rgb]{0.40,0.40,0.40}{##1}}}
\expandafter\def\csname PY@tok@kc\endcsname{\let\PY@bf=\textbf\def\PY@tc##1{\textcolor[rgb]{0.00,0.50,0.00}{##1}}}
\expandafter\def\csname PY@tok@c\endcsname{\let\PY@it=\textit\def\PY@tc##1{\textcolor[rgb]{0.25,0.50,0.50}{##1}}}
\expandafter\def\csname PY@tok@mf\endcsname{\def\PY@tc##1{\textcolor[rgb]{0.40,0.40,0.40}{##1}}}
\expandafter\def\csname PY@tok@err\endcsname{\def\PY@bc##1{\setlength{\fboxsep}{0pt}\fcolorbox[rgb]{1.00,0.00,0.00}{1,1,1}{\strut ##1}}}
\expandafter\def\csname PY@tok@mb\endcsname{\def\PY@tc##1{\textcolor[rgb]{0.40,0.40,0.40}{##1}}}
\expandafter\def\csname PY@tok@ss\endcsname{\def\PY@tc##1{\textcolor[rgb]{0.10,0.09,0.49}{##1}}}
\expandafter\def\csname PY@tok@sr\endcsname{\def\PY@tc##1{\textcolor[rgb]{0.73,0.40,0.53}{##1}}}
\expandafter\def\csname PY@tok@mo\endcsname{\def\PY@tc##1{\textcolor[rgb]{0.40,0.40,0.40}{##1}}}
\expandafter\def\csname PY@tok@kd\endcsname{\let\PY@bf=\textbf\def\PY@tc##1{\textcolor[rgb]{0.00,0.50,0.00}{##1}}}
\expandafter\def\csname PY@tok@mi\endcsname{\def\PY@tc##1{\textcolor[rgb]{0.40,0.40,0.40}{##1}}}
\expandafter\def\csname PY@tok@kn\endcsname{\let\PY@bf=\textbf\def\PY@tc##1{\textcolor[rgb]{0.00,0.50,0.00}{##1}}}
\expandafter\def\csname PY@tok@cpf\endcsname{\let\PY@it=\textit\def\PY@tc##1{\textcolor[rgb]{0.25,0.50,0.50}{##1}}}
\expandafter\def\csname PY@tok@kr\endcsname{\let\PY@bf=\textbf\def\PY@tc##1{\textcolor[rgb]{0.00,0.50,0.00}{##1}}}
\expandafter\def\csname PY@tok@s\endcsname{\def\PY@tc##1{\textcolor[rgb]{0.73,0.13,0.13}{##1}}}
\expandafter\def\csname PY@tok@kp\endcsname{\def\PY@tc##1{\textcolor[rgb]{0.00,0.50,0.00}{##1}}}
\expandafter\def\csname PY@tok@w\endcsname{\def\PY@tc##1{\textcolor[rgb]{0.73,0.73,0.73}{##1}}}
\expandafter\def\csname PY@tok@kt\endcsname{\def\PY@tc##1{\textcolor[rgb]{0.69,0.00,0.25}{##1}}}
\expandafter\def\csname PY@tok@sc\endcsname{\def\PY@tc##1{\textcolor[rgb]{0.73,0.13,0.13}{##1}}}
\expandafter\def\csname PY@tok@sb\endcsname{\def\PY@tc##1{\textcolor[rgb]{0.73,0.13,0.13}{##1}}}
\expandafter\def\csname PY@tok@k\endcsname{\let\PY@bf=\textbf\def\PY@tc##1{\textcolor[rgb]{0.00,0.50,0.00}{##1}}}
\expandafter\def\csname PY@tok@se\endcsname{\let\PY@bf=\textbf\def\PY@tc##1{\textcolor[rgb]{0.73,0.40,0.13}{##1}}}
\expandafter\def\csname PY@tok@sd\endcsname{\let\PY@it=\textit\def\PY@tc##1{\textcolor[rgb]{0.73,0.13,0.13}{##1}}}

\def\PYZbs{\char`\\}
\def\PYZus{\char`\_}
\def\PYZob{\char`\{}
\def\PYZcb{\char`\}}
\def\PYZca{\char`\^}
\def\PYZam{\char`\&}
\def\PYZlt{\char`\<}
\def\PYZgt{\char`\>}
\def\PYZsh{\char`\#}
\def\PYZpc{\char`\%}
\def\PYZdl{\char`\$}
\def\PYZhy{\char`\-}
\def\PYZsq{\char`\'}
\def\PYZdq{\char`\"}
\def\PYZti{\char`\~}
% for compatibility with earlier versions
\def\PYZat{@}
\def\PYZlb{[}
\def\PYZrb{]}
\makeatother


    % Exact colors from NB
    \definecolor{incolor}{rgb}{0.0, 0.0, 0.5}
    \definecolor{outcolor}{rgb}{0.545, 0.0, 0.0}



    
    % Prevent overflowing lines due to hard-to-break entities
    \sloppy 
    % Setup hyperref package
    \hypersetup{
      breaklinks=true,  % so long urls are correctly broken across lines
      colorlinks=true,
      urlcolor=blue,
      linkcolor=darkorange,
      citecolor=darkgreen,
      }
    % Slightly bigger margins than the latex defaults
    
    \geometry{verbose,tmargin=1in,bmargin=1in,lmargin=1in,rmargin=1in}
    
    

    \begin{document}
    
    
    \author{Claus-Dieter Ohl}\title{PH4606 - Lecture 6}

\date{\today}
\maketitle

    
    

    
\section{Medical Ultrasound 2}\label{medical-ultrasound-2}
\subsection{Bioeffects from ultrasound
imaging}\label{bioeffects-from-ultrasound-imaging}

Since 1992 ultrasound imaging systems are regulated to display two
indices to account for potential bioffects from heating and cavitation
(activation of gaseous nuclei). The ultrasonic scanners must show in
real time the thermal indidces (TIs) and the mechanical index (MI) which
is a function of the operation mode of the US scanner.

\subsection{Mechanical index}\label{mechanical-index}

The mechanical index, MI, in ultrasound imaging is defined as

\begin{equation}
\mathrm{MI}=\frac{\mathrm{PNP}}{\sqrt{f_c}}\label{eq:9.1}\tag{9.1}
\end{equation}

where PNP is the peak negative pressure at the position of imaging and
\(f_c\) is the center frequency of the imaging device.

For an MI \textless{} 0.3 the acoustic amplitude is considered low, for
0.3 \textless{} MI \textless{} 0.7 there is a possibility of minor
damage to neonatal lung or intestine. Above MI \textgreater{} 0.7 there
is a theoretical risk of cavitation. Medical US scanners are limited to
an MI of 1.9 for imaging.
\subsubsection{Rayleigh Plesset
Equation}\label{rayleigh-plesset-equation}

The Rayleigh Plesset Equation (without viscous dissipation) is given by
Eq. (9.2):

\begin{equation}
R \ddot{R} +\frac{3}{2} \dot{R}^2=\frac{p_v-p_\infty(t)}{\rho}+\frac{p_{g0}}{\rho}\left(\frac{R_0}{R}\right)^{3\kappa}-\frac{2\sigma}{\rho R}\label{9.2}\tag{9.2}
\end{equation}

The driving of the bubble occurs through the pressure \(p_\infty(t)\)
term. This could be for example through a harmonic driving
\(p_\infty(t)=p_0 + p_a \sin (\omega t)\) where \(\omega\) is the
angular frequency, \(p_0\) the static pressure, and \(p_a\) the
amplitude of the driving.

We want to solve this equation for small and large oscillations,
i.e.~for small and large pressures \(p_a\). To do so we first define the
constants for an air bubble in water at room temperature undergoing
isothermal compression.

The gas pressure at equilibrium is given by the equilibrium condition

\begin{equation}
p_v+p_{g0}=p_\infty+\frac{2\sigma}{R}\label{9.3}\tag{9.3}
\end{equation}

We can solve this nonlinear ODE with the built in solver in python.
Therefore, we have to rephrase the 2nd Order ODE into two first order
ODEs by variable substitution:

\begin{eqnarray}
y_0&=&R\\
y_1&=&\dot{R}
\end{eqnarray}

\begin{equation}
\dot{R}=\frac{dy_0}{dt}=y_1\label{9.4}\tag{9.4}
\end{equation}

and using the Rayleigh Equation we obtain:

\begin{equation}
\ddot{R}=\frac{dy_1}{dt}=\frac{1}{y_0}\left[
\frac{p_v-p_\infty(t)}{\rho}+\frac{p_g0}{\rho}\left(\frac{R_0}{y_0}\right)^{3\kappa}-\frac{2\sigma}{\rho y_0}-\frac{3}{2}(y_1)^2
\right]\label{9.5}\tag{9.5}
\end{equation}

We use a slightly modified version of the equation by introducing
additionally some dissipation from viscosity of the liquid. This term is
added on thr R.H.S of Eq. (9.5) which is \(-4\mu\dot{R}{R}^{-1}\).

These two first order ODEs are defined in the function \texttt{rp(t,y)}
\begin{Verbatim}[commandchars=\\\{\},gobble=2,numbers=left,fontsize=\small,baselinestretch=1]
\PY{o}{\PYZpc{}}\PY{k}{matplotlib} inline 
\PY{k+kn}{import} \PY{n+nn}{numpy} \PY{k+kn}{as} \PY{n+nn}{np} \PY{c+c1}{\PYZsh{}work with arrays}
\PY{k+kn}{from} \PY{n+nn}{scipy} \PY{k+kn}{import} \PY{n}{integrate}
\PY{c+c1}{\PYZsh{}from scipy.integrate import odeint}
\PY{k+kn}{from} \PY{n+nn}{matplotlib.pylab} \PY{k+kn}{import} \PY{o}{*}
\PY{k+kn}{from} \PY{n+nn}{IPython} \PY{k+kn}{import} \PY{n}{display}
\PY{k+kn}{from} \PY{n+nn}{ipywidgets} \PY{k+kn}{import} \PY{n}{widgets}

\PY{k}{global} \PY{n}{f}\PY{p}{,} \PY{n}{pa}\PY{p}{,} \PY{n}{R0}\PY{p}{,} \PY{n}{ncycles}

\PY{n}{pv}\PY{o}{=}\PY{l+m+mf}{2.3388e3}     \PY{c+c1}{\PYZsh{}[Pa] vapour pressure of water}
\PY{n}{sigma}\PY{o}{=}\PY{l+m+mf}{0.072}     \PY{c+c1}{\PYZsh{}[N/m] coefficient of surface tension between air and water}
\PY{n}{rho}\PY{o}{=}\PY{l+m+mf}{998.}        \PY{c+c1}{\PYZsh{}[kg/m\PYZca{}3] density of water}
\PY{n}{kappa}\PY{o}{=}\PY{l+m+mf}{1.}       \PY{c+c1}{\PYZsh{}polytropic exponent of the gas}
\PY{n}{p0}\PY{o}{=}\PY{l+m+mf}{1e5}          \PY{c+c1}{\PYZsh{}[Pa] static pressure}
\PY{n}{nu}\PY{o}{=}\PY{l+m+mf}{1e\PYZhy{}6}         \PY{c+c1}{\PYZsh{}[m\PYZca{}2/s]}

\PY{c+c1}{\PYZsh{}Rayleigh Plesset Equation}
\PY{k}{def} \PY{n+nf}{rp}\PY{p}{(}\PY{n}{t}\PY{p}{,}\PY{n}{y}\PY{p}{)}\PY{p}{:}
    \PY{k}{global} \PY{n}{f}\PY{p}{,} \PY{n}{pa}\PY{p}{,} \PY{n}{R0}\PY{p}{,} \PY{n}{ncycles}
    \PY{n}{pinfty}\PY{o}{=}\PY{n}{p0}\PY{o}{\PYZhy{}}\PY{n}{pa}\PY{o}{*}\PY{n}{np}\PY{o}{.}\PY{n}{sin}\PY{p}{(}\PY{n}{f}\PY{o}{*}\PY{l+m+mf}{2.}\PY{o}{*}\PY{n}{np}\PY{o}{.}\PY{n}{pi}\PY{o}{*}\PY{n}{t}\PY{p}{)}\PY{o}{*}\PY{p}{(}\PY{l+m+mf}{1.}\PY{o}{/}\PY{n}{f}\PY{o}{*}\PY{n}{ncycles}\PY{o}{\PYZgt{}}\PY{n}{t}\PY{p}{)}
    \PY{n}{pg0}\PY{o}{=}\PY{n}{p0}\PY{o}{+}\PY{l+m+mf}{2.}\PY{o}{*}\PY{n}{sigma}\PY{o}{/}\PY{n}{R0}\PY{o}{\PYZhy{}}\PY{n}{pv}
    \PY{n}{dydt0}\PY{o}{=}\PY{n}{y}\PY{p}{[}\PY{l+m+mi}{1}\PY{p}{]}
    \PY{n}{dydt1}\PY{o}{=}\PY{p}{(}\PY{n}{pv}\PY{o}{\PYZhy{}}\PY{n}{pinfty}\PY{o}{+}\PY{n}{pg0}\PY{o}{*}\PY{p}{(}\PY{n}{R0}\PY{o}{/}\PY{n}{y}\PY{p}{[}\PY{l+m+mi}{0}\PY{p}{]}\PY{p}{)}\PY{o}{*}\PY{o}{*}\PY{p}{(}\PY{l+m+mf}{3.}\PY{o}{*}\PY{n}{kappa}\PY{p}{)}\PYZbs{}
           \PY{o}{\PYZhy{}}\PY{l+m+mf}{2.}\PY{o}{*}\PY{n}{sigma}\PY{o}{/}\PY{n}{y}\PY{p}{[}\PY{l+m+mi}{0}\PY{p}{]}\PY{p}{)}\PY{o}{/}\PY{n}{rho}\PY{o}{/}\PY{n}{y}\PY{p}{[}\PY{l+m+mi}{0}\PY{p}{]}\PY{o}{\PYZhy{}}\PY{l+m+mf}{1.5}\PY{o}{*}\PY{n}{y}\PY{p}{[}\PY{l+m+mi}{1}\PY{p}{]}\PY{o}{*}\PY{o}{*}\PY{l+m+mf}{2.}\PY{o}{/}\PY{n}{y}\PY{p}{[}\PY{l+m+mi}{0}\PY{p}{]}\PYZbs{}
           \PY{o}{\PYZhy{}}\PY{l+m+mf}{4.}\PY{o}{*}\PY{n}{nu}\PY{o}{*}\PY{n}{y}\PY{p}{[}\PY{l+m+mi}{1}\PY{p}{]}\PY{o}{/}\PY{n}{y}\PY{p}{[}\PY{l+m+mi}{0}\PY{p}{]}\PY{o}{/}\PY{n}{y}\PY{p}{[}\PY{l+m+mi}{0}\PY{p}{]}
    \PY{k}{return} \PY{p}{[}\PY{n}{dydt0}\PY{p}{,}\PY{n}{dydt1}\PY{p}{]}

\PY{k}{def} \PY{n+nf}{plotall}\PY{p}{(}\PY{n}{newvalue}\PY{p}{)}\PY{p}{:}
    \PY{k}{global} \PY{n}{f}
    \PY{k}{global} \PY{n}{pa}
    \PY{k}{global} \PY{n}{R0}
    \PY{k}{global} \PY{n}{ncycles}
    
    \PY{c+c1}{\PYZsh{}set the variables according to the user input}
    \PY{n}{f}\PY{o}{=}\PY{n}{w\PYZus{}frequency}\PY{o}{.}\PY{n}{value}\PY{o}{*}\PY{l+m+mf}{1.e6}
    \PY{n}{pa}\PY{o}{=}\PY{n}{w\PYZus{}pressure}\PY{o}{.}\PY{n}{value}\PY{o}{*}\PY{l+m+mf}{1.e5}
    \PY{n}{R0}\PY{o}{=}\PY{n}{w\PYZus{}R0}\PY{o}{.}\PY{n}{value}\PY{o}{*}\PY{l+m+mf}{1.e\PYZhy{}6}
    \PY{n}{ncycles}\PY{o}{=}\PY{n+nb}{float}\PY{p}{(}\PY{n}{w\PYZus{}ncycles}\PY{o}{.}\PY{n}{value}\PY{p}{)}
    
    \PY{c+c1}{\PYZsh{}Setting initial conditions}
    \PY{n}{t\PYZus{}sim}\PY{o}{=}\PY{l+m+mf}{10.}\PY{o}{/}\PY{n}{f}
    \PY{n}{delta\PYZus{}t} \PY{o}{=} \PY{n}{t\PYZus{}sim}\PY{o}{/}\PY{l+m+mf}{1000.}
    \PY{n}{num\PYZus{}steps} \PY{o}{=} \PY{n+nb}{int}\PY{p}{(}\PY{n}{t\PYZus{}sim}\PY{o}{/}\PY{n}{delta\PYZus{}t}\PY{p}{)} \PY{o}{+} \PY{l+m+mi}{3}
    \PY{n}{time} \PY{o}{=} \PY{n}{np}\PY{o}{.}\PY{n}{zeros}\PY{p}{(}\PY{p}{(}\PY{n}{num\PYZus{}steps}\PY{p}{,} \PY{l+m+mi}{1}\PY{p}{)}\PY{p}{)}
    \PY{n}{R} \PY{o}{=} \PY{n}{np}\PY{o}{.}\PY{n}{zeros}\PY{p}{(}\PY{p}{(}\PY{n}{num\PYZus{}steps}\PY{p}{,} \PY{l+m+mi}{1}\PY{p}{)}\PY{p}{)}
    \PY{n}{time}\PY{p}{[}\PY{l+m+mi}{0}\PY{p}{]} \PY{o}{=} \PY{l+m+mf}{0.}
    \PY{n}{R}\PY{p}{[}\PY{l+m+mi}{0}\PY{p}{]} \PY{o}{=} \PY{n}{R0}

    \PY{n}{ode15s} \PY{o}{=} \PY{n}{integrate}\PY{o}{.}\PY{n}{ode}\PY{p}{(}\PY{n}{rp}\PY{p}{)}
    \PY{n}{ode15s}\PY{o}{.}\PY{n}{set\PYZus{}initial\PYZus{}value}\PY{p}{(}\PY{p}{[}\PY{n}{R}\PY{p}{[}\PY{l+m+mi}{0}\PY{p}{]}\PY{p}{,}\PY{l+m+mf}{0.}\PY{p}{]}\PY{p}{,} \PY{l+m+mf}{0.}\PY{p}{)}
    
    \PY{c+c1}{\PYZsh{}Integrate Rayleigh Plesset Equation and store the results every dt}
    \PY{n}{k} \PY{o}{=} \PY{l+m+mi}{1}
    \PY{k}{while} \PY{n}{ode15s}\PY{o}{.}\PY{n}{successful}\PY{p}{(}\PY{p}{)} \PY{o+ow}{and} \PY{n}{ode15s}\PY{o}{.}\PY{n}{t} \PY{o}{\PYZlt{}} \PY{p}{(}\PY{n}{t\PYZus{}sim}\PY{p}{)}\PY{p}{:}
        \PY{n}{ode15s}\PY{o}{.}\PY{n}{integrate}\PY{p}{(}\PY{n}{ode15s}\PY{o}{.}\PY{n}{t}\PY{o}{+}\PY{n}{delta\PYZus{}t}\PY{p}{)}
        \PY{n}{time}\PY{p}{[}\PY{n}{k}\PY{p}{]} \PY{o}{=} \PY{n}{ode15s}\PY{o}{.}\PY{n}{t}
        \PY{n}{R}\PY{p}{[}\PY{n}{k}\PY{p}{]} \PY{o}{=} \PY{n}{ode15s}\PY{o}{.}\PY{n}{y}\PY{p}{[}\PY{l+m+mi}{0}\PY{p}{]}
        \PY{n}{k} \PY{o}{+}\PY{o}{=} \PY{l+m+mi}{1}
        
    \PY{c+c1}{\PYZsh{}Plotting}
    \PY{n}{plt}\PY{o}{.}\PY{n}{figure}\PY{p}{(}\PY{l+m+mi}{1}\PY{p}{,} \PY{n}{figsize}\PY{o}{=}\PY{p}{(}\PY{l+m+mi}{10}\PY{p}{,} \PY{l+m+mi}{6}\PY{p}{)}\PY{p}{,} \PY{n}{dpi}\PY{o}{=}\PY{l+m+mi}{200}\PY{p}{)}
    \PY{n}{plt}\PY{o}{.}\PY{n}{clf}
    \PY{n}{plt}\PY{o}{.}\PY{n}{plot}\PY{p}{(}\PY{n}{time}\PY{p}{[}\PY{l+m+mi}{0}\PY{p}{:}\PY{n}{k}\PY{o}{\PYZhy{}}\PY{l+m+mi}{1}\PY{p}{]}\PY{o}{*}\PY{n}{f}\PY{p}{,}\PY{n}{R}\PY{p}{[}\PY{l+m+mi}{0}\PY{p}{:}\PY{n}{k}\PY{o}{\PYZhy{}}\PY{l+m+mi}{1}\PY{p}{]}\PY{o}{/}\PY{n}{R0}\PY{p}{)}\PY{p}{;}
    \PY{n}{plt}\PY{o}{.}\PY{n}{xlabel}\PY{p}{(}\PY{l+s+s1}{\PYZsq{}}\PY{l+s+s1}{Time (1/f)}\PY{l+s+s1}{\PYZsq{}}\PY{p}{)}
    \PY{n}{plt}\PY{o}{.}\PY{n}{ylabel}\PY{p}{(}\PY{l+s+s1}{\PYZsq{}}\PY{l+s+s1}{Radius (R0)}\PY{l+s+s1}{\PYZsq{}}\PY{p}{)}
    \PY{n}{plt}\PY{o}{.}\PY{n}{text}\PY{p}{(}\PY{n}{time}\PY{p}{[}\PY{n}{k}\PY{o}{\PYZhy{}}\PY{l+m+mi}{1}\PY{p}{]}\PY{o}{*}\PY{n}{f}\PY{o}{*}\PY{o}{.}\PY{l+m+mi}{8}\PY{p}{,}\PY{l+m+mf}{1.}\PY{p}{,}\PYZbs{}
             \PY{l+s+s1}{\PYZsq{}}\PY{l+s+s1}{MI=\PYZob{}0:.2f\PYZcb{}}\PY{l+s+s1}{\PYZsq{}}\PY{o}{.}\PY{n}{format}\PY{p}{(}\PY{n}{pa}\PY{o}{/}\PY{l+m+mf}{1e6}\PY{o}{/}\PY{n}{np}\PY{o}{.}\PY{n}{sqrt}\PY{p}{(}\PY{n}{f}\PY{o}{/}\PY{l+m+mf}{1e6}\PY{p}{)}\PY{p}{)}\PY{p}{,}\PYZbs{}
             \PY{n}{fontsize}\PY{o}{=}\PY{l+m+mi}{25}\PY{p}{,}\PY{n}{color}\PY{o}{=}\PY{l+s+s1}{\PYZsq{}}\PY{l+s+s1}{b}\PY{l+s+s1}{\PYZsq{}}\PY{p}{)}

    \PY{n}{display}\PY{o}{.}\PY{n}{clear\PYZus{}output}\PY{p}{(}\PY{n}{wait}\PY{o}{=}\PY{n+nb+bp}{True}\PY{p}{)}
    
\PY{c+c1}{\PYZsh{}Userinterface    }
\PY{n}{w\PYZus{}frequency}\PY{o}{=}\PY{n}{widgets}\PY{o}{.}\PY{n}{FloatSlider}\PY{p}{(}\PY{n+nb}{min}\PY{o}{=}\PY{o}{.}\PY{l+m+mi}{5}\PY{p}{,}\PY{n+nb}{max}\PY{o}{=}\PY{l+m+mi}{20}\PY{p}{,}\PY{n}{step}\PY{o}{=}\PY{o}{.}\PY{l+m+mi}{25}\PY{p}{,}\PYZbs{}
                                       \PY{n}{value}\PY{o}{=}\PY{l+m+mi}{5}\PY{p}{,}\PY{n}{description}\PY{o}{=}\PY{l+s+s1}{\PYZsq{}}\PY{l+s+s1}{Frequency (MHz):}\PY{l+s+s1}{\PYZsq{}}\PY{p}{)}
\PY{n}{w\PYZus{}pressure}\PY{o}{=}\PY{n}{widgets}\PY{o}{.}\PY{n}{FloatSlider}\PY{p}{(}\PY{n+nb}{min}\PY{o}{=}\PY{o}{.}\PY{l+m+mi}{1}\PY{p}{,}\PY{n+nb}{max}\PY{o}{=}\PY{l+m+mi}{30}\PY{p}{,}\PY{n}{step}\PY{o}{=}\PY{o}{.}\PY{l+m+mi}{1}\PY{p}{,}\PYZbs{}
                                       \PY{n}{value}\PY{o}{=}\PY{o}{.}\PY{l+m+mi}{4}\PY{p}{,}\PY{n}{description}\PY{o}{=}\PY{l+s+s1}{\PYZsq{}}\PY{l+s+s1}{Acoustic amplitude (bar):}\PY{l+s+s1}{\PYZsq{}}\PY{p}{)}
\PY{n}{w\PYZus{}R0}\PY{o}{=}\PY{n}{widgets}\PY{o}{.}\PY{n}{FloatSlider}\PY{p}{(}\PY{n+nb}{min}\PY{o}{=}\PY{o}{.}\PY{l+m+mi}{1}\PY{p}{,}\PY{n+nb}{max}\PY{o}{=}\PY{l+m+mi}{5}\PY{p}{,}\PY{n}{step}\PY{o}{=}\PY{o}{.}\PY{l+m+mi}{1}\PY{p}{,}\PYZbs{}
                                       \PY{n}{value}\PY{o}{=}\PY{o}{.}\PY{l+m+mi}{3}\PY{p}{,}\PY{n}{description}\PY{o}{=}\PY{l+s+s1}{\PYZsq{}}\PY{l+s+s1}{Nucleus size (micrometer):}\PY{l+s+s1}{\PYZsq{}}\PY{p}{)}
\PY{n}{w\PYZus{}ncycles}\PY{o}{=}\PY{n}{widgets}\PY{o}{.}\PY{n}{IntSlider}\PY{p}{(}\PY{n+nb}{min}\PY{o}{=}\PY{l+m+mi}{1}\PY{p}{,}\PY{n+nb}{max}\PY{o}{=}\PY{l+m+mi}{10}\PY{p}{,}\PY{n}{step}\PY{o}{=}\PY{l+m+mi}{1}\PY{p}{,}\PYZbs{}
                                       \PY{n}{value}\PY{o}{=}\PY{l+m+mi}{1}\PY{p}{,}\PY{n}{description}\PY{o}{=}\PY{l+s+s1}{\PYZsq{}}\PY{l+s+s1}{Number of cyles}\PY{l+s+s1}{\PYZsq{}}\PY{p}{)}
\PY{n}{w\PYZus{}frequency}\PY{o}{.}\PY{n}{observe}\PY{p}{(}\PY{n}{plotall}\PY{p}{,} \PY{n}{names}\PY{o}{=}\PY{l+s+s1}{\PYZsq{}}\PY{l+s+s1}{value}\PY{l+s+s1}{\PYZsq{}}\PY{p}{)} \PY{c+c1}{\PYZsh{}auto update}
\PY{n}{w\PYZus{}pressure}\PY{o}{.}\PY{n}{observe}\PY{p}{(}\PY{n}{plotall}\PY{p}{,} \PY{n}{names}\PY{o}{=}\PY{l+s+s1}{\PYZsq{}}\PY{l+s+s1}{value}\PY{l+s+s1}{\PYZsq{}}\PY{p}{)} 
\PY{n}{w\PYZus{}R0}\PY{o}{.}\PY{n}{observe}\PY{p}{(}\PY{n}{plotall}\PY{p}{,} \PY{n}{names}\PY{o}{=}\PY{l+s+s1}{\PYZsq{}}\PY{l+s+s1}{value}\PY{l+s+s1}{\PYZsq{}}\PY{p}{)}
\PY{n}{w\PYZus{}ncycles}\PY{o}{.}\PY{n}{observe}\PY{p}{(}\PY{n}{plotall}\PY{p}{,} \PY{n}{names}\PY{o}{=}\PY{l+s+s1}{\PYZsq{}}\PY{l+s+s1}{value}\PY{l+s+s1}{\PYZsq{}}\PY{p}{)}
\PY{n}{display}\PY{o}{.}\PY{n}{display}\PY{p}{(}\PY{n}{w\PYZus{}frequency}\PY{p}{)}
\PY{n}{display}\PY{o}{.}\PY{n}{display}\PY{p}{(}\PY{n}{w\PYZus{}pressure}\PY{p}{)}
\PY{n}{display}\PY{o}{.}\PY{n}{display}\PY{p}{(}\PY{n}{w\PYZus{}R0}\PY{p}{)}
\PY{n}{display}\PY{o}{.}\PY{n}{display}\PY{p}{(}\PY{n}{w\PYZus{}ncycles}\PY{p}{)}
\PY{n}{plotall}\PY{p}{(}\PY{l+m+mi}{1}\PY{p}{)}
    \end{Verbatim}

    \begin{figure}
        \begin{center}\adjustimage{max size={0.9\linewidth}{0.4\paperheight}}{lecture 9_files/lecture 9_4_0.png}\end{center}
        \caption{}
        \label{}
    \end{figure}
    \subsection{Thermal index}\label{thermal-index}

The thermal indes is defined as

\begin{equation}
\mathrm{TI}=\frac{W_0}{W_{deg}}\label{9.6}\tag{9.6}
\end{equation}

where \(W_0\) is the time-averaged acoustic power of the source and
\(W_{deg}\) is the power necessary to increase the target tissue by
\(1^\circ\)C. Three commonly used thermal indices are the thermal index
of soft tissue (TIS), the thermal index of bone (TIB) and the thermal
index of cranial bone (TIC). Recommendation is that at a TI value of 2.0
the examination should not exceed 60 min, while at a TI of 3.0 the
duration should be reduced to 15 min.
\subsection{High Intensity Focused
Ultrasound}\label{high-intensity-focused-ultrasound}

\texttt{\color{outcolor}Out[{\color{outcolor}10}]:}
    
    \begin{figure}
        \begin{center}\adjustimage{max size={0.9\linewidth}{0.4\paperheight}}{lecture 9_files/lecture 9_7_0.jpe}\end{center}
        \caption{}
        \label{}
    \end{figure}
    
A general introduction is available
\href{http://www.myvmc.com/treatments/high-intensity-focused-ultrasound-hifu}{here}.
\subsection{Histotripsy}\label{histotripsy}

\texttt{\color{outcolor}Out[{\color{outcolor}9}]:}
    
    \begin{figure}
        \begin{center}\adjustimage{max size={0.9\linewidth}{0.4\paperheight}}{lecture 9_files/lecture 9_10_0.jpe}\end{center}
        \caption{}
        \label{}
    \end{figure}
    
A general introduction is available \href{./pdfs/histotripsy.pdf}{here}
and a website with ample information is available
\href{http://www.histotripsy.umich.edu}{here}
\begin{Verbatim}[commandchars=\\\{\},gobble=2,numbers=left,fontsize=\small,baselinestretch=1]

    \end{Verbatim}


    % Add a bibliography block to the postdoc
    
    
    
    \end{document}


% Default to the notebook output style

    


% Inherit from the specified cell style.




    
\documentclass{article}
\usepackage{parskip}


\AtBeginDocument{
\heavyrulewidth=.08em
\lightrulewidth=.05em
\cmidrulewidth=.03em
\belowrulesep=.65ex
\belowbottomsep=0pt
\aboverulesep=.4ex
\abovetopsep=0pt
\cmidrulesep=\doublerulesep
\cmidrulekern=.5em
\defaultaddspace=.5em
}
\newcommand{\fudm}[2]{\frac{\mathrm{D} #1}{\mathrm{D} #2}}
\newcommand{\pad}[2]{\frac{\partial #1}{\partial #2}}
\newcommand{\ppad}[2]{\frac{\partial^2 #1}{\partial #2^2}}
\newcommand{\ppadd}[3]{\frac{\partial^2 #1}{\partial #2 \partial #3}}
\newcommand{\nnabla}{\nabla^2}
\newcommand{\eps}{\epsilon}
\newcommand{\vdetail}[1]{\vb{#1}=\begin{pmatrix}#1_1\\#1_2\\#1_3\end{pmatrix}}
\newcommand{\vb}[1]{\mathbf{#1}}
\newcommand{\va}[1]{\vec{#1}}
\newcommand{\tb}[1]{\underline{\underline{\mathbf{#1}}}}
\newcommand{\fud}[2]{\frac{\mathrm{d} #1}{\mathrm{d} #2}}



    
    
    \usepackage{graphicx} % Used to insert images
    \usepackage{adjustbox} % Used to constrain images to a maximum size 
    \usepackage{color} % Allow colors to be defined
    \usepackage{enumerate} % Needed for markdown enumerations to work
    \usepackage{geometry} % Used to adjust the document margins
    \usepackage{amsmath} % Equations
    \usepackage{amssymb} % Equations
    \usepackage{eurosym} % defines \euro
    \usepackage[mathletters]{ucs} % Extended unicode (utf-8) support
    \usepackage[utf8x]{inputenc} % Allow utf-8 characters in the tex document
    \usepackage{fancyvrb} % verbatim replacement that allows latex
    \usepackage{grffile} % extends the file name processing of package graphics 
                         % to support a larger range 
    % The hyperref package gives us a pdf with properly built
    % internal navigation ('pdf bookmarks' for the table of contents,
    % internal cross-reference links, web links for URLs, etc.)
    \usepackage{hyperref}
    \usepackage{longtable} % longtable support required by pandoc >1.10
    \usepackage{booktabs}  % table support for pandoc > 1.12.2
    \usepackage{ulem} % ulem is needed to support strikethroughs (\sout)
    

    
    
    \definecolor{orange}{cmyk}{0,0.4,0.8,0.2}
    \definecolor{darkorange}{rgb}{.71,0.21,0.01}
    \definecolor{darkgreen}{rgb}{.12,.54,.11}
    \definecolor{myteal}{rgb}{.26, .44, .56}
    \definecolor{gray}{gray}{0.45}
    \definecolor{lightgray}{gray}{.95}
    \definecolor{mediumgray}{gray}{.8}
    \definecolor{inputbackground}{rgb}{.95, .95, .85}
    \definecolor{outputbackground}{rgb}{.95, .95, .95}
    \definecolor{traceback}{rgb}{1, .95, .95}
    % ansi colors
    \definecolor{red}{rgb}{.6,0,0}
    \definecolor{green}{rgb}{0,.65,0}
    \definecolor{brown}{rgb}{0.6,0.6,0}
    \definecolor{blue}{rgb}{0,.145,.698}
    \definecolor{purple}{rgb}{.698,.145,.698}
    \definecolor{cyan}{rgb}{0,.698,.698}
    \definecolor{lightgray}{gray}{0.5}
    
    % bright ansi colors
    \definecolor{darkgray}{gray}{0.25}
    \definecolor{lightred}{rgb}{1.0,0.39,0.28}
    \definecolor{lightgreen}{rgb}{0.48,0.99,0.0}
    \definecolor{lightblue}{rgb}{0.53,0.81,0.92}
    \definecolor{lightpurple}{rgb}{0.87,0.63,0.87}
    \definecolor{lightcyan}{rgb}{0.5,1.0,0.83}
    
    % commands and environments needed by pandoc snippets
    % extracted from the output of `pandoc -s`
    \providecommand{\tightlist}{%
      \setlength{\itemsep}{0pt}\setlength{\parskip}{0pt}}
    \DefineVerbatimEnvironment{Highlighting}{Verbatim}{commandchars=\\\{\}}
    % Add ',fontsize=\small' for more characters per line
    \newenvironment{Shaded}{}{}
    \newcommand{\KeywordTok}[1]{\textcolor[rgb]{0.00,0.44,0.13}{\textbf{{#1}}}}
    \newcommand{\DataTypeTok}[1]{\textcolor[rgb]{0.56,0.13,0.00}{{#1}}}
    \newcommand{\DecValTok}[1]{\textcolor[rgb]{0.25,0.63,0.44}{{#1}}}
    \newcommand{\BaseNTok}[1]{\textcolor[rgb]{0.25,0.63,0.44}{{#1}}}
    \newcommand{\FloatTok}[1]{\textcolor[rgb]{0.25,0.63,0.44}{{#1}}}
    \newcommand{\CharTok}[1]{\textcolor[rgb]{0.25,0.44,0.63}{{#1}}}
    \newcommand{\StringTok}[1]{\textcolor[rgb]{0.25,0.44,0.63}{{#1}}}
    \newcommand{\CommentTok}[1]{\textcolor[rgb]{0.38,0.63,0.69}{\textit{{#1}}}}
    \newcommand{\OtherTok}[1]{\textcolor[rgb]{0.00,0.44,0.13}{{#1}}}
    \newcommand{\AlertTok}[1]{\textcolor[rgb]{1.00,0.00,0.00}{\textbf{{#1}}}}
    \newcommand{\FunctionTok}[1]{\textcolor[rgb]{0.02,0.16,0.49}{{#1}}}
    \newcommand{\RegionMarkerTok}[1]{{#1}}
    \newcommand{\ErrorTok}[1]{\textcolor[rgb]{1.00,0.00,0.00}{\textbf{{#1}}}}
    \newcommand{\NormalTok}[1]{{#1}}
    
    % Additional commands for more recent versions of Pandoc
    \newcommand{\ConstantTok}[1]{\textcolor[rgb]{0.53,0.00,0.00}{{#1}}}
    \newcommand{\SpecialCharTok}[1]{\textcolor[rgb]{0.25,0.44,0.63}{{#1}}}
    \newcommand{\VerbatimStringTok}[1]{\textcolor[rgb]{0.25,0.44,0.63}{{#1}}}
    \newcommand{\SpecialStringTok}[1]{\textcolor[rgb]{0.73,0.40,0.53}{{#1}}}
    \newcommand{\ImportTok}[1]{{#1}}
    \newcommand{\DocumentationTok}[1]{\textcolor[rgb]{0.73,0.13,0.13}{\textit{{#1}}}}
    \newcommand{\AnnotationTok}[1]{\textcolor[rgb]{0.38,0.63,0.69}{\textbf{\textit{{#1}}}}}
    \newcommand{\CommentVarTok}[1]{\textcolor[rgb]{0.38,0.63,0.69}{\textbf{\textit{{#1}}}}}
    \newcommand{\VariableTok}[1]{\textcolor[rgb]{0.10,0.09,0.49}{{#1}}}
    \newcommand{\ControlFlowTok}[1]{\textcolor[rgb]{0.00,0.44,0.13}{\textbf{{#1}}}}
    \newcommand{\OperatorTok}[1]{\textcolor[rgb]{0.40,0.40,0.40}{{#1}}}
    \newcommand{\BuiltInTok}[1]{{#1}}
    \newcommand{\ExtensionTok}[1]{{#1}}
    \newcommand{\PreprocessorTok}[1]{\textcolor[rgb]{0.74,0.48,0.00}{{#1}}}
    \newcommand{\AttributeTok}[1]{\textcolor[rgb]{0.49,0.56,0.16}{{#1}}}
    \newcommand{\InformationTok}[1]{\textcolor[rgb]{0.38,0.63,0.69}{\textbf{\textit{{#1}}}}}
    \newcommand{\WarningTok}[1]{\textcolor[rgb]{0.38,0.63,0.69}{\textbf{\textit{{#1}}}}}
    
    
    % Define a nice break command that doesn't care if a line doesn't already
    % exist.
    \def\br{\hspace*{\fill} \\* }
    % Math Jax compatability definitions
    \def\gt{>}
    \def\lt{<}
    % Document parameters
    \title{lecture 8}
    
    
    

    % Pygments definitions
    
\makeatletter
\def\PY@reset{\let\PY@it=\relax \let\PY@bf=\relax%
    \let\PY@ul=\relax \let\PY@tc=\relax%
    \let\PY@bc=\relax \let\PY@ff=\relax}
\def\PY@tok#1{\csname PY@tok@#1\endcsname}
\def\PY@toks#1+{\ifx\relax#1\empty\else%
    \PY@tok{#1}\expandafter\PY@toks\fi}
\def\PY@do#1{\PY@bc{\PY@tc{\PY@ul{%
    \PY@it{\PY@bf{\PY@ff{#1}}}}}}}
\def\PY#1#2{\PY@reset\PY@toks#1+\relax+\PY@do{#2}}

\expandafter\def\csname PY@tok@gd\endcsname{\def\PY@tc##1{\textcolor[rgb]{0.63,0.00,0.00}{##1}}}
\expandafter\def\csname PY@tok@gu\endcsname{\let\PY@bf=\textbf\def\PY@tc##1{\textcolor[rgb]{0.50,0.00,0.50}{##1}}}
\expandafter\def\csname PY@tok@gt\endcsname{\def\PY@tc##1{\textcolor[rgb]{0.00,0.27,0.87}{##1}}}
\expandafter\def\csname PY@tok@gs\endcsname{\let\PY@bf=\textbf}
\expandafter\def\csname PY@tok@gr\endcsname{\def\PY@tc##1{\textcolor[rgb]{1.00,0.00,0.00}{##1}}}
\expandafter\def\csname PY@tok@cm\endcsname{\let\PY@it=\textit\def\PY@tc##1{\textcolor[rgb]{0.25,0.50,0.50}{##1}}}
\expandafter\def\csname PY@tok@vg\endcsname{\def\PY@tc##1{\textcolor[rgb]{0.10,0.09,0.49}{##1}}}
\expandafter\def\csname PY@tok@vi\endcsname{\def\PY@tc##1{\textcolor[rgb]{0.10,0.09,0.49}{##1}}}
\expandafter\def\csname PY@tok@mh\endcsname{\def\PY@tc##1{\textcolor[rgb]{0.40,0.40,0.40}{##1}}}
\expandafter\def\csname PY@tok@cs\endcsname{\let\PY@it=\textit\def\PY@tc##1{\textcolor[rgb]{0.25,0.50,0.50}{##1}}}
\expandafter\def\csname PY@tok@ge\endcsname{\let\PY@it=\textit}
\expandafter\def\csname PY@tok@vc\endcsname{\def\PY@tc##1{\textcolor[rgb]{0.10,0.09,0.49}{##1}}}
\expandafter\def\csname PY@tok@il\endcsname{\def\PY@tc##1{\textcolor[rgb]{0.40,0.40,0.40}{##1}}}
\expandafter\def\csname PY@tok@go\endcsname{\def\PY@tc##1{\textcolor[rgb]{0.53,0.53,0.53}{##1}}}
\expandafter\def\csname PY@tok@cp\endcsname{\def\PY@tc##1{\textcolor[rgb]{0.74,0.48,0.00}{##1}}}
\expandafter\def\csname PY@tok@gi\endcsname{\def\PY@tc##1{\textcolor[rgb]{0.00,0.63,0.00}{##1}}}
\expandafter\def\csname PY@tok@gh\endcsname{\let\PY@bf=\textbf\def\PY@tc##1{\textcolor[rgb]{0.00,0.00,0.50}{##1}}}
\expandafter\def\csname PY@tok@ni\endcsname{\let\PY@bf=\textbf\def\PY@tc##1{\textcolor[rgb]{0.60,0.60,0.60}{##1}}}
\expandafter\def\csname PY@tok@nl\endcsname{\def\PY@tc##1{\textcolor[rgb]{0.63,0.63,0.00}{##1}}}
\expandafter\def\csname PY@tok@nn\endcsname{\let\PY@bf=\textbf\def\PY@tc##1{\textcolor[rgb]{0.00,0.00,1.00}{##1}}}
\expandafter\def\csname PY@tok@no\endcsname{\def\PY@tc##1{\textcolor[rgb]{0.53,0.00,0.00}{##1}}}
\expandafter\def\csname PY@tok@na\endcsname{\def\PY@tc##1{\textcolor[rgb]{0.49,0.56,0.16}{##1}}}
\expandafter\def\csname PY@tok@nb\endcsname{\def\PY@tc##1{\textcolor[rgb]{0.00,0.50,0.00}{##1}}}
\expandafter\def\csname PY@tok@nc\endcsname{\let\PY@bf=\textbf\def\PY@tc##1{\textcolor[rgb]{0.00,0.00,1.00}{##1}}}
\expandafter\def\csname PY@tok@nd\endcsname{\def\PY@tc##1{\textcolor[rgb]{0.67,0.13,1.00}{##1}}}
\expandafter\def\csname PY@tok@ne\endcsname{\let\PY@bf=\textbf\def\PY@tc##1{\textcolor[rgb]{0.82,0.25,0.23}{##1}}}
\expandafter\def\csname PY@tok@nf\endcsname{\def\PY@tc##1{\textcolor[rgb]{0.00,0.00,1.00}{##1}}}
\expandafter\def\csname PY@tok@si\endcsname{\let\PY@bf=\textbf\def\PY@tc##1{\textcolor[rgb]{0.73,0.40,0.53}{##1}}}
\expandafter\def\csname PY@tok@s2\endcsname{\def\PY@tc##1{\textcolor[rgb]{0.73,0.13,0.13}{##1}}}
\expandafter\def\csname PY@tok@nt\endcsname{\let\PY@bf=\textbf\def\PY@tc##1{\textcolor[rgb]{0.00,0.50,0.00}{##1}}}
\expandafter\def\csname PY@tok@nv\endcsname{\def\PY@tc##1{\textcolor[rgb]{0.10,0.09,0.49}{##1}}}
\expandafter\def\csname PY@tok@s1\endcsname{\def\PY@tc##1{\textcolor[rgb]{0.73,0.13,0.13}{##1}}}
\expandafter\def\csname PY@tok@ch\endcsname{\let\PY@it=\textit\def\PY@tc##1{\textcolor[rgb]{0.25,0.50,0.50}{##1}}}
\expandafter\def\csname PY@tok@m\endcsname{\def\PY@tc##1{\textcolor[rgb]{0.40,0.40,0.40}{##1}}}
\expandafter\def\csname PY@tok@gp\endcsname{\let\PY@bf=\textbf\def\PY@tc##1{\textcolor[rgb]{0.00,0.00,0.50}{##1}}}
\expandafter\def\csname PY@tok@sh\endcsname{\def\PY@tc##1{\textcolor[rgb]{0.73,0.13,0.13}{##1}}}
\expandafter\def\csname PY@tok@ow\endcsname{\let\PY@bf=\textbf\def\PY@tc##1{\textcolor[rgb]{0.67,0.13,1.00}{##1}}}
\expandafter\def\csname PY@tok@sx\endcsname{\def\PY@tc##1{\textcolor[rgb]{0.00,0.50,0.00}{##1}}}
\expandafter\def\csname PY@tok@bp\endcsname{\def\PY@tc##1{\textcolor[rgb]{0.00,0.50,0.00}{##1}}}
\expandafter\def\csname PY@tok@c1\endcsname{\let\PY@it=\textit\def\PY@tc##1{\textcolor[rgb]{0.25,0.50,0.50}{##1}}}
\expandafter\def\csname PY@tok@o\endcsname{\def\PY@tc##1{\textcolor[rgb]{0.40,0.40,0.40}{##1}}}
\expandafter\def\csname PY@tok@kc\endcsname{\let\PY@bf=\textbf\def\PY@tc##1{\textcolor[rgb]{0.00,0.50,0.00}{##1}}}
\expandafter\def\csname PY@tok@c\endcsname{\let\PY@it=\textit\def\PY@tc##1{\textcolor[rgb]{0.25,0.50,0.50}{##1}}}
\expandafter\def\csname PY@tok@mf\endcsname{\def\PY@tc##1{\textcolor[rgb]{0.40,0.40,0.40}{##1}}}
\expandafter\def\csname PY@tok@err\endcsname{\def\PY@bc##1{\setlength{\fboxsep}{0pt}\fcolorbox[rgb]{1.00,0.00,0.00}{1,1,1}{\strut ##1}}}
\expandafter\def\csname PY@tok@mb\endcsname{\def\PY@tc##1{\textcolor[rgb]{0.40,0.40,0.40}{##1}}}
\expandafter\def\csname PY@tok@ss\endcsname{\def\PY@tc##1{\textcolor[rgb]{0.10,0.09,0.49}{##1}}}
\expandafter\def\csname PY@tok@sr\endcsname{\def\PY@tc##1{\textcolor[rgb]{0.73,0.40,0.53}{##1}}}
\expandafter\def\csname PY@tok@mo\endcsname{\def\PY@tc##1{\textcolor[rgb]{0.40,0.40,0.40}{##1}}}
\expandafter\def\csname PY@tok@kd\endcsname{\let\PY@bf=\textbf\def\PY@tc##1{\textcolor[rgb]{0.00,0.50,0.00}{##1}}}
\expandafter\def\csname PY@tok@mi\endcsname{\def\PY@tc##1{\textcolor[rgb]{0.40,0.40,0.40}{##1}}}
\expandafter\def\csname PY@tok@kn\endcsname{\let\PY@bf=\textbf\def\PY@tc##1{\textcolor[rgb]{0.00,0.50,0.00}{##1}}}
\expandafter\def\csname PY@tok@cpf\endcsname{\let\PY@it=\textit\def\PY@tc##1{\textcolor[rgb]{0.25,0.50,0.50}{##1}}}
\expandafter\def\csname PY@tok@kr\endcsname{\let\PY@bf=\textbf\def\PY@tc##1{\textcolor[rgb]{0.00,0.50,0.00}{##1}}}
\expandafter\def\csname PY@tok@s\endcsname{\def\PY@tc##1{\textcolor[rgb]{0.73,0.13,0.13}{##1}}}
\expandafter\def\csname PY@tok@kp\endcsname{\def\PY@tc##1{\textcolor[rgb]{0.00,0.50,0.00}{##1}}}
\expandafter\def\csname PY@tok@w\endcsname{\def\PY@tc##1{\textcolor[rgb]{0.73,0.73,0.73}{##1}}}
\expandafter\def\csname PY@tok@kt\endcsname{\def\PY@tc##1{\textcolor[rgb]{0.69,0.00,0.25}{##1}}}
\expandafter\def\csname PY@tok@sc\endcsname{\def\PY@tc##1{\textcolor[rgb]{0.73,0.13,0.13}{##1}}}
\expandafter\def\csname PY@tok@sb\endcsname{\def\PY@tc##1{\textcolor[rgb]{0.73,0.13,0.13}{##1}}}
\expandafter\def\csname PY@tok@k\endcsname{\let\PY@bf=\textbf\def\PY@tc##1{\textcolor[rgb]{0.00,0.50,0.00}{##1}}}
\expandafter\def\csname PY@tok@se\endcsname{\let\PY@bf=\textbf\def\PY@tc##1{\textcolor[rgb]{0.73,0.40,0.13}{##1}}}
\expandafter\def\csname PY@tok@sd\endcsname{\let\PY@it=\textit\def\PY@tc##1{\textcolor[rgb]{0.73,0.13,0.13}{##1}}}

\def\PYZbs{\char`\\}
\def\PYZus{\char`\_}
\def\PYZob{\char`\{}
\def\PYZcb{\char`\}}
\def\PYZca{\char`\^}
\def\PYZam{\char`\&}
\def\PYZlt{\char`\<}
\def\PYZgt{\char`\>}
\def\PYZsh{\char`\#}
\def\PYZpc{\char`\%}
\def\PYZdl{\char`\$}
\def\PYZhy{\char`\-}
\def\PYZsq{\char`\'}
\def\PYZdq{\char`\"}
\def\PYZti{\char`\~}
% for compatibility with earlier versions
\def\PYZat{@}
\def\PYZlb{[}
\def\PYZrb{]}
\makeatother


    % Exact colors from NB
    \definecolor{incolor}{rgb}{0.0, 0.0, 0.5}
    \definecolor{outcolor}{rgb}{0.545, 0.0, 0.0}



    
    % Prevent overflowing lines due to hard-to-break entities
    \sloppy 
    % Setup hyperref package
    \hypersetup{
      breaklinks=true,  % so long urls are correctly broken across lines
      colorlinks=true,
      urlcolor=blue,
      linkcolor=darkorange,
      citecolor=darkgreen,
      }
    % Slightly bigger margins than the latex defaults
    
    \geometry{verbose,tmargin=1in,bmargin=1in,lmargin=1in,rmargin=1in}
    
    

    \begin{document}
    
    
    \author{Claus-Dieter Ohl}\title{PH4606 - Lecture 6}

\date{\today}
\maketitle

    
    

    
\section{Medical Imaging}\label{medical-imaging}
\begin{Verbatim}[commandchars=\\\{\},gobble=2,numbers=left,fontsize=\small,baselinestretch=1]
\PY{k+kn}{from} \PY{n+nn}{IPython.display} \PY{k+kn}{import} \PY{n}{YouTubeVideo}
\PY{n}{YouTubeVideo}\PY{p}{(}\PY{l+s+s2}{\PYZdq{}}\PY{l+s+s2}{hH7FspvTI9I}\PY{l+s+s2}{\PYZdq{}}\PY{p}{,}\PY{n}{width}\PY{o}{=}\PY{l+m+mi}{800}\PY{p}{,} \PY{n}{height}\PY{o}{=}\PY{l+m+mi}{600}\PY{p}{)}
    \end{Verbatim}
\texttt{\color{outcolor}Out[{\color{outcolor}4}]:}
    
    \begin{figure}
        \begin{center}\adjustimage{max size={0.9\linewidth}{0.4\paperheight}}{lecture 8_files/lecture 8_2_0.jpe}\end{center}
        \caption{}
        \label{}
    \end{figure}
    
Modern ultrasonic scanners resemble personal computers with hard drives,
internet connection and integrated peripheral devices. They are
connected to ultrasonic probes.

Figure 8.1: Picture of a diagnostic ultrasound device.

Figure 8.2: Ultrasound scanner with a) Monitor, b) Manual controls, c)
several probes, d) backup on DVD, e) printer.

Figure 8.3: Functional blocks of an US scanner
\begin{Verbatim}[commandchars=\\\{\},gobble=2,numbers=left,fontsize=\small,baselinestretch=1]
\PY{n}{YouTubeVideo}\PY{p}{(}\PY{l+s+s2}{\PYZdq{}}\PY{l+s+s2}{JqVGgq5bE\PYZhy{}Y}\PY{l+s+s2}{\PYZdq{}}\PY{p}{,}\PY{n}{width}\PY{o}{=}\PY{l+m+mi}{800}\PY{p}{,} \PY{n}{height}\PY{o}{=}\PY{l+m+mi}{600}\PY{p}{)}
    \end{Verbatim}
\texttt{\color{outcolor}Out[{\color{outcolor}5}]:}
    
    \begin{figure}
        \begin{center}\adjustimage{max size={0.9\linewidth}{0.4\paperheight}}{lecture 8_files/lecture 8_4_0.jpe}\end{center}
        \caption{}
        \label{}
    \end{figure}
    
\subsection{A-mode scanning}\label{a-mode-scanning}

A single beam display methode, the A-mode where A stands for amplitude,
is the simplest form of a scanner. The scanner receives the reflection
from a single burst of ultrasound and displays it along a line on a
display.

Figure 8.4: A-mode scanning of for an example an a) eye. b) radio
frequency (RF) data, c) Hilbert transformed data b) Manual controls, c)
several probes, d) backup on DVD, e) printer.

Consider that the speed of sound in the medium is not changing we can
relate the time axis with the depth axis \(z\), e.g.

\begin{equation}
d=\frac{c t}{2 PRF}
\end{equation}

\(d\) is the maximum depth, \(PRF\) the pulse repetition frequency, and
the factor 2 comes from forth and back propagation of the beam. The most
commonly used mean diagnostic ultrasound speed is \(c=1540\,\)m/s. Thus
to increase the depth the \(PRF\) has to be decreased.

Figure 8.4 depicts the steps in generating the A mode picture:

\begin{itemize}
\item
  \begin{enumerate}
  \def\labelenumi{(\alph{enumi})}
  \tightlist
  \item
    orientation of the US scan head
  \end{enumerate}
\item
  \begin{enumerate}
  \def\labelenumi{(\alph{enumi})}
  \setcounter{enumi}{1}
  \tightlist
  \item
    radio frequency data from the scan head
  \end{enumerate}
\item
  \begin{enumerate}
  \def\labelenumi{(\alph{enumi})}
  \setcounter{enumi}{2}
  \tightlist
  \item
    envelope of the RF calculated with the Hilbert transform (see code
    below)
  \end{enumerate}
\item
  \begin{enumerate}
  \def\labelenumi{(\alph{enumi})}
  \setcounter{enumi}{3}
  \tightlist
  \item
    time gain compensation (TGC) to account for geometrical and
    absorption losses.
  \end{enumerate}
\item
  \begin{enumerate}
  \def\labelenumi{(\alph{enumi})}
  \setcounter{enumi}{4}
  \tightlist
  \item
    A-mode diagram
  \end{enumerate}
\item
  \begin{enumerate}
  \def\labelenumi{(\alph{enumi})}
  \setcounter{enumi}{5}
  \tightlist
  \item
    1-dimensional B-mode
  \end{enumerate}
\end{itemize}

The Hilbert transform allows to calculate the remove a fast oscillating
frequency (the RF data) from a signal and keep the slowly varying
envelope. Some information on the Hilbert transform is
\href{http://complextoreal.com/wp-content/uploads/2013/01/tcomplex.pdf}{given
here} and an example code below.

Your turn: * Study the program. * Identify which signal is representing
the RF data and which the depth signal * Would higher carrier
frequencies improve the A mode signal? Explain.
\begin{Verbatim}[commandchars=\\\{\},gobble=2,numbers=left,fontsize=\small,baselinestretch=1]
\PY{o}{\PYZpc{}}\PY{k}{matplotlib} inline
\PY{k+kn}{import} \PY{n+nn}{math} \PY{k+kn}{as} \PY{n+nn}{m}
\PY{k+kn}{import} \PY{n+nn}{numpy} \PY{k+kn}{as} \PY{n+nn}{np}
\PY{k+kn}{import} \PY{n+nn}{matplotlib.pyplot} \PY{k+kn}{as} \PY{n+nn}{plt}
\PY{k+kn}{import} \PY{n+nn}{scipy.signal} \PY{k+kn}{as} \PY{n+nn}{sig} \PY{c+c1}{\PYZsh{}this contains the Hilbert transform}
\PY{k+kn}{from} \PY{n+nn}{ipywidgets} \PY{k+kn}{import} \PY{n}{widgets}

\PY{k}{def} \PY{n+nf}{pltenvelope}\PY{p}{(}\PY{n}{f1}\PY{p}{,}\PY{n}{f2}\PY{p}{)}\PY{p}{:}
    \PY{n}{t}\PY{o}{=}\PY{n}{np}\PY{o}{.}\PY{n}{linspace}\PY{p}{(}\PY{l+m+mf}{0.}\PY{p}{,}\PY{l+m+mf}{5.}\PY{p}{,}\PY{l+m+mi}{500}\PY{p}{)}
    \PY{n}{y}\PY{o}{=}\PY{n}{np}\PY{o}{.}\PY{n}{sin}\PY{p}{(}\PY{n}{f1}\PY{o}{*}\PY{n}{t}\PY{o}{*}\PY{n}{np}\PY{o}{.}\PY{n}{pi}\PY{p}{)}\PY{o}{*}\PY{n}{np}\PY{o}{.}\PY{n}{sin}\PY{p}{(}\PY{n}{f2}\PY{o}{*}\PY{n}{t}\PY{o}{*}\PY{n}{np}\PY{o}{.}\PY{n}{pi}\PY{p}{)}
    \PY{n}{plt}\PY{o}{.}\PY{n}{plot}\PY{p}{(}\PY{n}{t}\PY{p}{,}\PY{n}{y}\PY{p}{)}
    \PY{n}{hilbert}\PY{o}{=}\PY{n}{np}\PY{o}{.}\PY{n}{imag}\PY{p}{(}\PY{n}{sig}\PY{o}{.}\PY{n}{hilbert}\PY{p}{(}\PY{n}{y}\PY{p}{)}\PY{p}{)}
    \PY{n}{plt}\PY{o}{.}\PY{n}{plot}\PY{p}{(}\PY{n}{t}\PY{p}{,}\PY{n}{np}\PY{o}{.}\PY{n}{sqrt}\PY{p}{(}\PY{n}{y}\PY{o}{*}\PY{o}{*}\PY{l+m+mi}{2}\PY{o}{+}\PY{n}{hilbert}\PY{o}{*}\PY{o}{*}\PY{l+m+mi}{2}\PY{p}{)}\PY{p}{)}\PY{p}{;}
    \PY{n}{plt}\PY{o}{.}\PY{n}{show}\PY{p}{(}\PY{p}{)}

\PY{n}{widgets}\PY{o}{.}\PY{n}{interact}\PY{p}{(}\PY{n}{pltenvelope}\PY{p}{,}\PYZbs{}
                 \PY{n}{f1}\PY{o}{=}\PY{n}{widgets}\PY{o}{.}\PY{n}{FloatSlider}\PY{p}{(}\PY{n+nb}{min}\PY{o}{=}\PY{l+m+mf}{1.}\PY{p}{,}\PY{n+nb}{max}\PY{o}{=}\PY{l+m+mf}{20.}\PY{p}{,}\PY{n}{step}\PY{o}{=}\PY{o}{.}\PY{l+m+mi}{25}\PY{p}{,}\PY{n}{value}\PY{o}{=}\PY{l+m+mf}{10.}\PY{p}{,}\PY{n}{description}\PY{o}{=}\PY{l+s+s1}{\PYZsq{}}\PY{l+s+s1}{carrier frequency}\PY{l+s+s1}{\PYZsq{}}\PY{p}{)}\PY{p}{,}\PYZbs{}
                 \PY{n}{f2}\PY{o}{=}\PY{n}{widgets}\PY{o}{.}\PY{n}{FloatSlider}\PY{p}{(}\PY{n+nb}{min}\PY{o}{=}\PY{o}{.}\PY{l+m+mi}{1}\PY{p}{,}\PY{n+nb}{max}\PY{o}{=}\PY{l+m+mi}{1}\PY{p}{,}\PY{n}{step}\PY{o}{=}\PY{o}{.}\PY{l+m+mo}{05}\PY{p}{,}\PY{n}{value}\PY{o}{=}\PY{o}{.}\PY{l+m+mi}{3}\PY{p}{,}\PY{n}{description}\PY{o}{=}\PY{l+s+s1}{\PYZsq{}}\PY{l+s+s1}{modulation frequency}\PY{l+s+s1}{\PYZsq{}}\PY{p}{)}\PY{p}{)}\PY{p}{;}
    \end{Verbatim}

    \begin{figure}
        \begin{center}\adjustimage{max size={0.9\linewidth}{0.4\paperheight}}{lecture 8_files/lecture 8_6_0.png}\end{center}
        \caption{}
        \label{}
    \end{figure}
    \subsection{B-mode scanning}\label{b-mode-scanning}

If in the image the pixels are displayed with the brightness
corresponding to the strength of the reflected signal we obtain the
B-mode display where B stands for brightness. Here, the brightness is
also adjusted with the TGC. Typically B mode images are 2d images where
the azimuthal axis of the scan head is along the horizontal axis and the
depth axis on the vertical.

Figure 8.5: B-mode image.

Your turn: * Try to identify object(s) which may cause some trouble in
the B-mode scan in Fig. 8.5 * Estimate a typical frame rates for B-mode
scanning
\subsection{M-mode scanning}\label{m-mode-scanning}

In M-mode (M stands for motion) the echos from a single beam direction
is displayed as a function of time on the horizontal. This has the
advantage that at very high time resolution is achieved, up to 1kHz
which allows to visualize the motion of a heart valves.

Figure 8.6: Sketch of M-mode working principle.

Figure 8.7: Example of the M-mode working principle.
\subsection{Beamsteering}\label{beamsteering}

Figure 8.8 shows how we can obtain controlled steering of the beam by
using six crystal elements during transmission with a short time delay
between separate transmission pulses. A typical number of elements is
64.

Figure 8.8: Electronic beamsteering with 6 elements.

Figure 8.8a depicts a constant phase leading to a plane wave travelling
normal to the scan head, while Fig. 8.8b introduces a fixed delay
between neighboring transducers.

Your turn * Design a scan head that has a single lobe within
\(\pm 30^\circ\) with 64 transducers operating at \(4\,\)MHz. What range
of time delays to you need? What is the maximum framing rate in B-mode
if you want to resolve 32 aximuthal lines and a depth of \(100\,\)mm.

\subsection{Examples of scan heads}\label{examples-of-scan-heads}

Figure 8.9: Scanheads a) linear array for 11MHz, b) curvelinear array at
4MHz, and 5MHz 1.5D matrix phase array.
\subsection{Doppler Methods}\label{doppler-methods}

If the sound source is emitting at a frequency \(f\) is moving with a
velocity \(v_s\) towrads the audience at rest, the wavelength is reduced
to

\begin{equation}
\lambda'=\frac{c-v_s}{f}
\end{equation}

Hence the frequency measured is

\begin{equation}
f'=\frac{f}{1-\frac{v_s}{c}}
\end{equation}

If the source moves away from the receiver \(v_s\) is negative.

If the receiver is moving at a velocity \(v_r\) towards the sound
source, the frequency experienced is

\begin{equation}
f'=\frac{c+v_r}{\lambda}=\left(1+\frac{v_r}{c}\right) f
\end{equation}

The frequency detected from the US scan head for a scattering object
moving relative to the scan head contains both contributions. The
frequency shift \(f_D=f'-f\) is called the Doppler shift. This can be
btained from above equations as

\begin{equation}
f_D=\frac{2\frac{v}{c}\cos\Theta}{1-\frac{v}{c}\cos \Theta}\,f
\end{equation}

where v is the magnitude of the velocity of the blood, c is the speed
sound, and \(\Theta\) the angle between the US beam and the blood flow
direction. Since \(c\ll v\)

\begin{equation}
f_D=2 \, f \,\frac{v}{c} \cos\Theta
\end{equation}

The doppler shift is positive if \(f_D\) is towards the US probe and
negative if it is away.

Below you can explore the dependency of the Doppler frequency from the
angle \(\Theta\) and the scanner frequency:
\begin{Verbatim}[commandchars=\\\{\},gobble=2,numbers=left,fontsize=\small,baselinestretch=1]
\PY{o}{\PYZpc{}}\PY{k}{matplotlib} inline
\PY{k+kn}{import} \PY{n+nn}{math} \PY{k+kn}{as} \PY{n+nn}{m}
\PY{k+kn}{import} \PY{n+nn}{numpy} \PY{k+kn}{as} \PY{n+nn}{np}
\PY{k+kn}{import} \PY{n+nn}{matplotlib.pyplot} \PY{k+kn}{as} \PY{n+nn}{plt}
\PY{k+kn}{from} \PY{n+nn}{ipywidgets} \PY{k+kn}{import} \PY{n}{widgets}

\PY{k}{def} \PY{n+nf}{pltdoppler}\PY{p}{(}\PY{n}{theta}\PY{p}{,}\PY{n}{f}\PY{p}{)}\PY{p}{:}
    \PY{n}{f}\PY{o}{=}\PY{n}{f}\PY{o}{*}\PY{l+m+mf}{1e6}
    \PY{n}{c}\PY{o}{=}\PY{l+m+mf}{1540.}
    \PY{n}{v}\PY{o}{=}\PY{n}{np}\PY{o}{.}\PY{n}{linspace}\PY{p}{(}\PY{l+m+mf}{0.}\PY{p}{,}\PY{l+m+mf}{5.}\PY{p}{,}\PY{l+m+mi}{500}\PY{p}{)}
    \PY{n}{vcos}\PY{o}{=}\PY{n}{v}\PY{o}{*}\PY{n}{np}\PY{o}{.}\PY{n}{cos}\PY{p}{(}\PY{n}{theta}\PY{o}{*}\PY{n}{np}\PY{o}{.}\PY{n}{pi}\PY{o}{/}\PY{l+m+mf}{180.}\PY{p}{)}
    \PY{n}{vcosoc}\PY{o}{=}\PY{n}{vcos}\PY{o}{/}\PY{n}{c}
    \PY{n}{fd}\PY{o}{=}\PY{l+m+mi}{2}\PY{o}{*}\PY{n}{vcosoc}\PY{o}{/}\PY{p}{(}\PY{l+m+mi}{1}\PY{o}{\PYZhy{}}\PY{n}{vcosoc}\PY{p}{)}\PY{o}{*}\PY{n}{f}
    \PY{n}{plt}\PY{o}{.}\PY{n}{plot}\PY{p}{(}\PY{n}{v}\PY{p}{,}\PY{n}{fd}\PY{p}{)}
    \PY{n}{plt}\PY{o}{.}\PY{n}{xlabel}\PY{p}{(}\PY{l+s+s1}{\PYZsq{}}\PY{l+s+s1}{v (m/s)}\PY{l+s+s1}{\PYZsq{}}\PY{p}{)}
    \PY{n}{plt}\PY{o}{.}\PY{n}{ylabel}\PY{p}{(}\PY{l+s+s1}{\PYZsq{}}\PY{l+s+s1}{f Doppler (Hz)}\PY{l+s+s1}{\PYZsq{}}\PY{p}{)}
    \PY{n}{plt}\PY{o}{.}\PY{n}{show}\PY{p}{(}\PY{p}{)}

\PY{n}{widgets}\PY{o}{.}\PY{n}{interact}\PY{p}{(}\PY{n}{pltdoppler}\PY{p}{,}\PYZbs{}
                 \PY{n}{theta}\PY{o}{=}\PY{n}{widgets}\PY{o}{.}\PY{n}{FloatSlider}\PY{p}{(}\PY{n+nb}{min}\PY{o}{=}\PY{l+m+mf}{0.}\PY{p}{,}\PY{n+nb}{max}\PY{o}{=}\PY{l+m+mi}{90}\PY{p}{,}\PY{n}{step}\PY{o}{=}\PY{l+m+mi}{5}\PY{p}{,}\PY{n}{value}\PY{o}{=}\PY{l+m+mf}{30.}\PY{p}{,}\PY{n}{description}\PY{o}{=}\PY{l+s+s1}{\PYZsq{}}\PY{l+s+s1}{Theta}\PY{l+s+s1}{\PYZsq{}}\PY{p}{)}\PY{p}{,}\PYZbs{}
                 \PY{n}{f}\PY{o}{=}\PY{n}{widgets}\PY{o}{.}\PY{n}{FloatSlider}\PY{p}{(}\PY{n+nb}{min}\PY{o}{=}\PY{o}{.}\PY{l+m+mi}{5}\PY{p}{,}\PY{n+nb}{max}\PY{o}{=}\PY{l+m+mi}{15}\PY{p}{,}\PY{n}{step}\PY{o}{=}\PY{o}{.}\PY{l+m+mi}{5}\PY{p}{,}\PY{n}{value}\PY{o}{=}\PY{l+m+mi}{4}\PY{p}{,}\PY{n}{description}\PY{o}{=}\PY{l+s+s1}{\PYZsq{}}\PY{l+s+s1}{frequency in MHz}\PY{l+s+s1}{\PYZsq{}}\PY{p}{)}\PY{p}{)}\PY{p}{;}
    \end{Verbatim}

    \begin{figure}
        \begin{center}\adjustimage{max size={0.9\linewidth}{0.4\paperheight}}{lecture 8_files/lecture 8_11_0.png}\end{center}
        \caption{}
        \label{}
    \end{figure}
    \subsubsection{Single-beam Doppler
methods}\label{single-beam-doppler-methods}

Here a the Doppler shift is registered from one Doppler beam. The
Doppler signal is typically in the audible range and can be listen to
via a loudspeaker.

Figure 8.10: a) Continuous wave Doppler (CWD), b) pulsed wave Doppler
(PWD), c) high pulse repetition frequency Doppler (HPRF).

\subsubsection{Continuous wave Doppler}\label{continuous-wave-doppler}

Figure 8.11: Continuous wave Doppler (CWD), note the measurement of high
velocities of up to 4 m/s.

In continuous wave Doppler (CDW) mode the transmit elements send a
continuous ultrasound signal and a separate receiver element detects the
reflected and Doppler shifted signal. The CWD mode is sensitive to the
grayed area in Fig. 8.10 a).

\subsubsection{Pulsed wave Doppler}\label{pulsed-wave-doppler}

Figure 8.12: Pulsed wave Doppler (PWD).

In pulsed wave Doppler (PWD) mode the transmitting and receiving
elements are the same as used for example in B-mode imaging. A short
pulse of about 6-12 cycles is transmitted with a pulse repetition
frequency of 5-15kHz. The receiver is opened (gated) for a short time
window to limit the sample volume, see Fig. 8.10. Because the frequency
resolution is tight to the sampling frequency and the Doppler frequency
is sampled with the PRF the maximum frequency which can be measured is

\begin{equation}
f_{D,max}=\frac{PRF}{2} \quad ,
\end{equation}

where the denominator is due to
\href{https://en.wikipedia.org/wiki/Nyquist–Shannon_sampling_theorem}{Nyquist
theorem}.

The maximum PRF is related to the depth \(R_{max}\) range the maximum
velocity one can measure with PWD mode is

\begin{equation}
v_{max}=\frac{c^2}{8 f_0 R_{max}} \quad ,
\end{equation}

Your turn * Derive above expression.

\subsubsection{High pulse repetition frequency Doppler
(HPRF)}\label{high-pulse-repetition-frequency-doppler-hprf}

Here the next pulse is transmitted before the echoes from the first
sample volume have been registered. Here the radiologist needs to be
careful with the placement of the depth volume. Figure 8.10 c) depicts
HPRF Doppler where only inside the artery the velocity is sampled.

\subsubsection{Color Doppler}\label{color-doppler}

With many sample volumes along the beam many Doppler scan line processor
channels are needed in parallel. Sweeping the beam with multiple sample
volumes gives velocity information in a 2D area. The mean velocity is
the converted to a color code and overlayed on the grayscale B-mode
image. This method is called color Doppler (CD). Typically the
resolution for color Doppler is lower than for B-mode as a larger beam
width (sample volume) is required for getting sufficient echoe strength
from moving blood. Color Doppler may be affected already at normal
arterial velocities from the Nyquist limit.

Figure 8.13: Color Doppler of the liver.


    % Add a bibliography block to the postdoc
    
    
    
    \end{document}
